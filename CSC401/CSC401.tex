\documentclass[11pt]{article}
\usepackage{fullpage}
\usepackage{amsmath}
\usepackage{mathtools}
\usepackage{esint}
\usepackage{cancel}
\usepackage{graphicx}
\usepackage{xcolor}
\usepackage{float}
\linespread{1.2}
\allowdisplaybreaks
\usepackage{color}
\usepackage{listings}
\usepackage{subfigure}
\usepackage{multicol}
\usepackage{xcolor}
\usepackage{sectsty}
\definecolor{darkblue}{RGB}{10,0,100}
\definecolor{otherblue}{RGB}{0,70,200}
\sectionfont{\color{darkblue}} 
\subsectionfont{\color{otherblue}}  


\begin{document}

\title{CSC401  \\ Natural Language Computing}
\author{Michael Boyadjian}
\maketitle
\pagebreak

\tableofcontents

\pagebreak

\bigskip
\bigskip
\bigskip


\section{Introduction to Natural Language Computing}
\hrule \vspace{15pt}

Natural language computing is getting computers to understand everything we say and write.  We are interested in learning the statistics of language.  Computers
give insight into how humans process language, or generate language themselves.
\subsection{Categories of Linguistic Knowledge}
\begin{itemize}
\item \textbf{Phonology}: The study of patterns of speech sounds
\item \textbf{Morphology}: How words can be changed by inflection or derivation
\item \textbf{Syntax}: The ordering and structure between words and phrases (i.e. \textit{grammar})
\item \textbf{Semantics}: The study of how meaning is created by words and phrases
\item \textbf{Pragmatics}: The study of meaning in contexts
\end{itemize}

\subsection{NLP as Artificial Intelligence}
Natural language processing involves resolving ambiguity at all levels. In the early days knowledge was explicitly encoded in artificial symbolic systems by experts. Now, algorithms learn using probabilities (or pseudo-probabilities)
to distinguish subtly different competing hypotheses.
\begin{itemize}
\item Is \textbf{Google} a noun or a verb?
\item Examples where $Google \in Nouns$ (“\textit{Google makes Android”}),
does not mean that $Google$ is never a verb (\textit{“Go Google yourself”})
\end{itemize}
$$ P(Google \in Nouns) > P(Google \in Verbs) > 0 $$

\subsection{Overview of NLP}
Is natural language processing (the discipline) hard?
\begin{itemize}
\item \textbf{Yes}, because natural language
\begin{itemize}
\item is highly ambiguous at all levels
\item is complex and subtle
\item is fuzzy and probabilistic
\item involves real-world reasoning
\end{itemize}
\item \textbf{No}, because computer science
\begin{itemize}
\item gives us many powerful statistical techniques,
\item allows us to break the challenges down into more
manageable features.
\end{itemize}

\end{itemize}



\pagebreak
\section{Corpora and Smoothing}
\section{Features and Classification}
\section{Neural Language Models}
\section{Hidden Markov Models}
\section{Statistical Machine Translation}
\section{Speech}
\section{Automatic Speech Recognition}
\section{Synthesis}
\section{Entropy Decisions}
\section{Intelligent Dialogue Agents}


\end{document}